\documentclass{bsuir}

\departmentlong{инженерной психологии и эргономики}
\workcode{2}
\worktitle{Использование языка программирования Swift:\\функции, замыкания, перечисления}
\titleleft{
    Проверил:\\
    Усенко Ф.В.\\
    ~
}
\titleright{
    Выполнил:\\
    Бородин А.Н.\\
    гр. 310901
}
\titlepageyear{2024}

\begin{document}

\maketitle
\mainmatter
\renewcommand{\thefigure}{\arabic{figure}}
\renewcommand{\thelisting}{\arabic{listing}}

\textbf{Цель}: Выполнить разработку приложения с использованием языка
программирования Swift: функции, замыкания, перечисления.

\section*{Задание 1}

Среди всех четырехзначных номеров машин, определите количество номеров,
содержащих только три одинаковые цифры.

\makelisting{Sources/01/main.swift}[main.swift]

\makelisting{1.txt}[Вывод программы]

\section*{Задание 2}

Используя функциональные типы, создайте программу:

\begin{itemize}
    \item для умножения целых чисел;
    \item для умножения комплексных чисел.
\end{itemize}

\makelisting{Sources/02/main.swift}[main.swift]

\makelisting{2.txt}[Вывод программы]

\textbf{Вывод}: освоен базис языка программирования swift.

\end{document}
