\documentclass{bsuir}

\departmentlong{инженерной психологии и эргономики}
\workcode{3}
\worktitle{Использование языка программирования Swift:\\структуры и классы, методы}
\titleleft{
    Проверил:\\
    Усенко Ф.В.\\
    ~
}
\titleright{
    Выполнил:\\
    Бородин А.Н.\\
    гр. 310901
}
\titlepageyear{2024}

\begin{document}

\maketitle
\mainmatter
\renewcommand{\thefigure}{\arabic{figure}}
\renewcommand{\thelisting}{\arabic{listing}}

\textbf{Цель}: Выполнить разработку приложения с использованием языка
программирования Swift: структуры и классы, методы.

\section*{Задание 1}

Создайте класс ПЕРСОНА с методами, что позволяет вывести на экран информацию о
персоне, а также определить её возраст (в текущем году). Создайте дочерние
классы: АБИТУРИЕНТ (фамилия, дата рождения, факультет), СТУДЕНТ (фамилия, дата
рождения, факультет, курс), ПРЕПОДАВАТЕЛЬ (фамилия, дата рождения, факультет,
должность, стаж), со своими методами вывода информации на экран и определения
возраста. Создайте список из n персон, выведите полную информацию из базы на
экран, а также организуйте поиск персон, чей возраст попадает в заданный
диапазон.

\makelisting{Sources/01/main.swift}[main.swift]

\makelisting{1.txt}[Вывод программы]

\textbf{Вывод}: освоены классы и структуры в языке программирования swift.

\end{document}
