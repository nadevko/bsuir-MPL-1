\documentclass{bsuir}

\departmentlong{инженерной психологии и эргономики}
\workcode{1}
\worktitle{Использование языка программирования Kotlin}
\titleleft{
    Проверил:\\
    Усенко Ф.В.\\
    ~
}
\titleright{
    Выполнил:\\
    Бородин А.Н.\\
    гр. 310901
}
\titlepageyear{2024}

\begin{document}

\maketitle
\mainmatter
\renewcommand{\thefigure}{\arabic{figure}}
\renewcommand{\thelisting}{\arabic{listing}}

\textbf{Цель}: Изучить синтаксис и основную логику языка kotlin.

\section*{Задание 3}

Реализовать программу, описанную диаграммами. Дополнительно, написать тесты.

\makeimage[Диаграмма классов]{classes.png}[width=.9\textwidth]

\makeimage[Диаграмма последовательности]{sequence.png}[width=.9\textwidth]

\makelisting{src/main/kotlin/Main.kt}[Main.kt]

\makelisting{src/test/kotlin/MainTest.kt}[MainTest.kt]

\makelisting{1.txt}[Вывод программы]

\textbf{Вывод}: освоен базис языка программирования kotlin.

\end{document}
